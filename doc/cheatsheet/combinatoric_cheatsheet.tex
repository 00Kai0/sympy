\documentclass[10pt,landscape]{article}
\usepackage{multicol}
\usepackage{calc}
\usepackage{ifthen}
\usepackage[landscape]{geometry}

% To make this come out properly in landscape mode, do one of the following
% 1.
%  pdflatex latexsheet.tex
%
% 2.
%  latex latexsheet.tex
%  dvips -P pdf  -t landscape latexsheet.dvi
%  ps2pdf latexsheet.ps


% If you're reading this, be prepared for confusion.  Making this was
% a learning experience for me, and it shows.  Much of the placement
% was hacked in; if you make it better, let me know...


% 2008-04
% Changed page margin code to use the geometry package. Also added code for
% conditional page margins, depending on paper size. Thanks to Uwe Ziegenhagen
% for the suggestions.

% 2006-08
% Made changes based on suggestions from Gene Cooperman. <gene at ccs.neu.edu>


% To Do:
% \listoffigures \listoftables
% \setcounter{secnumdepth}{0}


% This sets page margins to .5 inch if using letter paper, and to 1cm
% if using A4 paper. (This probably isn't strictly necessary.)
% If using another size paper, use default 1cm margins.
\ifthenelse{\lengthtest { \paperwidth = 11in}}
	{ \geometry{top=.5in,left=.5in,right=.5in,bottom=.5in} }
	{\ifthenelse{ \lengthtest{ \paperwidth = 297mm}}
		{\geometry{top=1cm,left=1cm,right=1cm,bottom=1cm} }
		{\geometry{top=1cm,left=1cm,right=1cm,bottom=1cm} }
	}

% Turn off header and footer
\pagestyle{empty}


% Redefine section commands to use less space
\makeatletter
\renewcommand{\section}{\@startsection{section}{1}{0mm}%
                                {-1ex plus -.5ex minus -.2ex}%
                                {0.5ex plus .2ex}%x
                                {\normalfont\large\bfseries}}
\renewcommand{\subsection}{\@startsection{subsection}{2}{0mm}%
                                {-1explus -.5ex minus -.2ex}%
                                {0.5ex plus .2ex}%
                                {\normalfont\normalsize\bfseries}}
\renewcommand{\subsubsection}{\@startsection{subsubsection}{3}{0mm}%
                                {-1ex plus -.5ex minus -.2ex}%
                                {1ex plus .2ex}%
                                {\normalfont\small\bfseries}}
\makeatother

% Define BibTeX command
\def\BibTeX{{\rm B\kern-.05em{\sc i\kern-.025em b}\kern-.08em
    T\kern-.1667em\lower.7ex\hbox{E}\kern-.125emX}}

% Don't print section numbers
\setcounter{secnumdepth}{0}


\setlength{\parindent}{0pt}
\setlength{\parskip}{0pt plus 0.5ex}


% -----------------------------------------------------------------------
   \newcommand{\tab}{\hspace*{2em}}

\begin{document}

\raggedright
\footnotesize
\begin{multicols}{3}


% multicol parameters
% These lengths are set only within the two main columns
%\setlength{\columnseprule}{0.25pt}
\setlength{\premulticols}{1pt}
\setlength{\postmulticols}{1pt}
\setlength{\multicolsep}{1pt}
\setlength{\columnsep}{2pt}

\begin{center}
     \Large{\textbf{Combinatoric CheatSheet \\ Sympy.org}} \\
\end{center}
\section{Partition}
\verb!Path: from sympy.combinatorics.partitions!\\
\subsection{Methods}
\verb!random_integer_partition(n, seed=None)! \\
Generates a random integer partition summing to n as a list of reverse-sorted integers


\verb!RGS_generalized(m)! \\
Computes the \verb!m + 1! generalized unrestricted growth strings and returns them as rows in matrix


\verb!RGS_enum(m)! \\
computes the total number of restricted growth
 strings possible for a superset of size \verb!m!


\verb!RGS_unrank(rank, m)! \\
Gives the unranked restricted growth string for a given superset size


\verb!RGS_rank(rgs)! \\
Computes the rank of a restricted growth string.


\subsection{Subclass \texttt{Partition}}
A partition is a set of disjoint sets whose union equals a given set.
This Class represent abstract partition.

\begin{tabular}{@{}lp{4.5cm}l@{}}
\verb!rgs!    & Restricted Growth String\\
\verb!from_rgs(rgs,elements)!  & Creates a set partition from a RSG\\
\verb!rank! & Gets the rank of a partition \\
\verb!partition!  & Return partition as a sorted list of lists\\
\verb!sort_key(order=None)!  & Return a canonical key that can be used for sorting.
\end{tabular}

%dump : \\
%Used at the very beginning of a document:
%\verb!\documentclass{!\textit{class}\verb!}!.  Use
%\verb!\begin{document}! to start contents and \verb!\end{document}! to
%end the document.
%\\end dump.

\subsection{Subclass \texttt{IntegerPartition}}
This class represents an integer partition.
\newlength{\MyLen}
\settowidth{\MyLen}{\texttt{letterpaper}/\texttt{a4paper} \ }
\begin{tabular}{@{}p{\the\MyLen}%
                @{}p{\linewidth-\the\MyLen}@{}}
\verb!as_dict()! & Return the partition as a dictionary whose keys are the
				partition integers and the values are the multiplicity of that integer \\
\verb!as_ferrers(char='#')! & Prints the ferrer diagram of a partition\\
\verb!conjugate! &Computes the conjugate partition of itself\\
\verb!next_lex()! & Return the next partition of the integer, n, in lexical order\\
\verb!prev_lex()! & Return the previous partition of the integer, n, in lexical order\\
\end{tabular}

\section{Permutation}
Path \verb!sympy.combinatorics.permutations.Permutation!\\
\subsection{Methods}
\verb!array_form! \\
This is used to convert from cyclic notation to the canonical notation

\verb!ascents()!\\
Returns the positions of ascents in a permutation, i.e., the location where $p[i] < p[i+1]$


\verb!descents()!
Returns the positions of descents in a permutation, i.e., the location where $p[i] > p[i+1]$


\verb!atoms()!\\
Returns all the elements of a permutation

\verb!cardinality!\\
Returns the number of all possible permutations.

\verb!commutator(x)!\\
Return the commutator of self and x: \verb!~x*~self*x*self!


\verb!commutes_with(other)!\\
Checks if the elements are commuting.


\verb!cycle_structure!\\
Return the cycle structure of the permutation as a dictionary indicating the multiplicity of each cycle length.


\verb!cycles! \\
Returns the number of cycles contained in the permutation (including singletons).

\verb!cyclic_form!\\
This is used to convert to the cyclic notation from the canonical notation. Singletons are omitted.

\verb!from_inversion_vector(inversion)!\\
Calculates the permutation from the inversion vector.


\verb!from_sequence(i, key=None)!\\
Return the permutation needed to obtain \verb!i! from the sorted elements of \verb!i!.
If custom sorting is desired, a key can be given.

\verb!full_cyclic_form!\\
Return permutation in cyclic form including singletons.


\verb!get_adjacency_distance(other)!\\
Computes the adjacency distance between two permutations.


\verb!get_adjacency_matrix()!\\
Computes the adjacency matrix of a permutation.


\verb!get_positional_distance(other)!\\
Computes the positional distance between two permutations.


\verb!get_precedence_distance(other)!\\
Computes the precedence distance between two permutations.


\verb!get_precedence_matrix()!\\
Gets the precedence matrix. This is used for computing the distance between two permutations.


\verb!index()!\\
Returns the index of a permutation.


\verb!inversion_vector()!\\
Return the inversion vector of the permutation.


\verb!inversions()!\\
Computes the number of inversions of a permutation.


\verb!is_Empty!\\
Checks to see if the permutation is a set with zero elements


\verb!is_Identity!\\
Returns True if the Permutation is an identity permutation.


\verb!is_Singleton!\\
Checks to see if the permutation contains only one number and
 is thus the only possible permutation of this set of numbers.


 \verb!is_even!\\
Checks if a permutation is even.


\verb!is_odd!\\
Checks if a permutation is odd.


\verb!josephus(m, n, s=1)!\\
Return as a permutation the shuffling of range(\verb!n!) using the Josephus scheme
in which every \verb!m!-th item is selected until all have been chosen.


\verb!length()!\\
Returns the number of integers moved by a permutation.


\verb!list(size=None)!\\
Return the permutation as an explicit list


\verb!max()!\\
The maximum element moved by the permutation.


\verb!min()!\\
The minimum element moved by the permutation


\verb!next_lex()!\\
Returns the next permutation in lexicographical order.


\verb!next_nonlex()!\\
Returns the next permutation in nonlex order.


\verb!next_trotterjohnson()!\\
Returns the next permutation in Trotter-Johnson order.


\verb!order()!\\
Computes the order of a permutation.


\verb!parity()!\\
Computes the parity of a permutation.


\verb!random(n)!\\
Generates a random permutation of length \verb!n!.


\verb!rank(i=None)!\\
Returns the lexicographic rank of the permutation
(default) or the \verb!i!th ranked permutation of self.


\verb!rank_nonlex(inv_perm=None)!\\
This is a linear time ranking algorithm
that does not enforce lexicographic order.


\verb!rank_trotterjohnson()!\\
Returns the Trotter Johnson rank, which we get from the minimal change algorithm.


\verb!static rmul(*args)!\\
Return product of Permutations $[a, b, c, \ldots]$ as the Permutation whose $i$th value is $a(b(c(i)))$.


\verb!runs()!\\
Returns the runs of a permutation.


\verb!signature()!\\
Gives the signature of the permutation needed to place the elements
 of the permutation in canonical order.


\verb!size!\\
Returns the number of elements in the permutation.


\verb!support()!\\
Return the elements in permutation, $P$, for which $P[i] \neq i$.



\verb!transpositions()!\\
Return the permutation decomposed into a list of transpositions.


\verb!unrank_lex(size, rank)!\\
Lexicographic permutation unranking.



\verb!unrank_nonlex(n, r)!\\
This is a linear time unranking algorithm that does not respect lexicographic order.


\verb!unrank_trotterjohnson(size, rank)!\\
Trotter Johnson permutation unranking.

\subsection{Subclass \texttt{Cycle(*args)}}
Wrapper around dict which provides the functionality of a disjoint cycle.


\subsection{Subclass \texttt{Generators}}

\verb!symmetric(n)!\\
Generates the symmetric group of order \verb!n!, \verb!Sn!.


\verb!cyclic(n)!\\
Generates the cyclic group of order \verb!n, Cn!.


\verb!alternating(n)!\\
Generates the alternating group of order \verb!n, An!


\verb!dihedral(n)!\\
Generates the dihedral group of order \verb!2n, Dn!.


\section{PermutationGroup}
Path : \verb!sympy.combinatorics.perm_groups.PermutationGroup!


\subsection{Methods}
\verb!base!\\
Return a base from the Schreier-Sims algorithm.

\verb!baseswap(base, strong_gens, pos, randomized=False,!\\
\verb!transversals=None, basic_orbits=None,! \\
\verb!strong_gens_distr=None)!\\
Swap two consecutive base points in base and strong generating set.



\verb!basic_orbits!\\
Return the basic orbits relative to a base and strong generating set.



\verb!basic_stabilizers!\\
Return a chain of stabilizers relative to a base and strong generating set.



\verb!basic_transversals!\\
Return basic transversals relative to a base and strong generating set.



\verb!center()!\\
Return the center of a permutation group.



\verb!centralizer(other)!\\
Return the centralizer of a group/set/element.



\verb!commutator(G, H)!\\
Return the commutator of two subgroups.


\verb!contains(g, strict=True)!\\
Test if permutation \verb!g! belong to self.



\verb!coset_factor(g, af=False)!\\
Return G�s (self�s) coset factorization, \verb!f,! of \verb!g!.




\verb!coset_rank(g)!\\
rank using Schreier-Sims representation



\verb!coset_unrank(rank, af=False)!\\
unrank using Schreier-Sims representation



\verb!degree!\\
Returns the size of the permutations in the group.



\verb!derived_series()!\\
Return the derived series for the group.


\verb!derived_subgroup()!\\
Compute the derived subgroup.


\verb!generate(method='coset', af=False)!\\
Return iterator to generate the elements of the group


\verb!generate_dimino(af=False)!\\
Yield group elements using Dimino�s algorithm


\verb!generate_schreier_sims(af=False)!\\
Yield group elements using the Schreier-Sims representation.



\verb!generators!\\
Returns the generators of the group.




\verb!is_abelian!\\
Test if the group is Abelian.


\verb!is_alt_sym(eps=0.05, _random_prec=None)!\\
Monte Carlo test for the symmetric/alternating group for degrees $>= 8$.



\verb!is_group()!\\
Return True if the group if identity is present, the inverse of every element is also an element, and the product of any two elements is also an element.



\verb!is_nilpotent!\\
Test if the group is nilpotent.


\verb!is_normal(gr)!\\
Test if G=self is a normal subgroup of \verb!gr!.


\verb!is_primitive(randomized=True)!\\
Test if a group is primitive.


\verb!is_solvable!\\
Test if the group is solvable.


\verb!is_subgroup(G, strict=True)!\\
Return True if all elements of self belong to G.



\verb!is_transitive(strict=True)!\\
Test if the group is transitive.


\verb!is_trivial!\\
Test if the group is the trivial group.


\verb!lower_central_series()! \\
Return the lower central series for the group.


\verb!make_perm(n, seed=None)!\\
Multiply n randomly selected permutations from pgroup together, starting with the identity permutation.



\verb!max_div!\\
Maximum proper divisor of the degree of a permutation group.



\verb!minimal_block(points)!\\
For a transitive group, finds the block system generated by \verb!points!.



\verb!normal_closure(other, k=10)!\\
Return the normal closure of a subgroup/set of permutations.


\verb!orbit(alpha, action='tuples')!\\
Compute the orbit of alpha \verb!\{g(\alpha) | g \in G\}! as a set.


\verb!orbit_rep(alpha, beta, schreier_vector=None)!\\
Return a group element which sends \verb!alpha! to \verb!beta!.


\verb!orbit_transversal(alpha, pairs=False)!\\
Computes a transversal for the orbit of \verb!alpha! as a set.


\verb!orbits(rep=False)!\\
Return the orbits of self, ordered according to lowest element in each orbit.


\verb!order()!\\
Return the number of permutations that can be generated from elements of the group.



\verb!pointwise_stabilizer(points, incremental=False)!\\
Return the pointwise stabilizer for a set of points.



\verb!random(af=False)!\\
Return a random group element.



\verb!random_pr(gen_count=11, iterations=50, _random_prec=None)!\\
Return a random group element using product replacement.


\verb!random_stab(alpha, schreier_vector=None, _random_prec=None)!\\
Random element from the stabilizer of \verb!alpha!.


\verb!schreier_sims()!\\
Schreier-Sims algorithm.



\verb!schreier_sims_incremental(base=None, gens=None)!\\
Extend a sequence of points and generating set to a base and strong generating set.



\verb!schreier_sims_random(base=None, gens=None,!\\



\verb! consec_succ=10, _random_prec=None)!\\
Randomized Schreier-Sims algorithm.



\verb!schreier_vector(alpha)! \\
Computes the schreier vector for \verb!alpha!.


\verb!stabilizer(alpha)!\\
Return the stabilizer subgroup of \verb!alpha!.


\verb!stabilizer_cosets(af=False)!\\
Return a list of cosets of the stabilizer chain of the group as computed by the Schreir-Sims algorithm.



\verb!stabilizer_gens(af=False)!\\
Return the generators of the chain of stabilizers of the Schreier-Sims representation.



\verb!strong_gens!\\
Return a strong generating set from the Schreier-Sims algorithm.



\verb!subgroup_search(prop, base=None, strong_gens=None,!\\



\verb!tests=None, init_subgroup=None)!\\
Find the subgroup of all elements satisfying the property prop.


\verb!transitivity_degree!\\
Compute the degree of transitivity of the group.


\section{Polyhedron}
Path : \verb!sympy.combinatorics.polyhedron.Polyhedron!\\
Represents the polyhedral symmetry group (PSG).
\subsection{Methods}
\verb!array_form!\\
Return the indices of the corners.


\verb!corners!\\
Get the corners of the Polyhedron.



\verb!cyclic_form!\\
Return the indices of the corners in cyclic notation.


\verb!edges!\\
Given the faces of the polyhedra we can get the edges.


\verb!faces!\\
Get the faces of the Polyhedron.



\verb!pgroup!\\
Get the permutations of the Polyhedron.



\verb!reset()!\\
Return corners to their original positions.



\verb!rotate(perm)!\\
Apply a permutation to the polyhedron in place.


\verb!size!\\
Get the number of corners of the Polyhedron.

\verb!vertices!\\
Get the corners of the Polyhedron.

\section{Prufer}
Path: \verb!sympy.combinatorics.prufer.Prufer!\\
The Prufer correspondence is an algorithm that describes the bijection
 between labeled trees and the Prufer code.
  A Prufer code of a labeled tree is unique up to
   isomorphism and has a length of $n - 2$.
\subsection{Methods}


\verb!static edges(*runs)!\\
Return a list of edges and the number of nodes from the given
 runs that connect nodes in an integer-labelled tree.

 \verb!next(delta=1)!\\
Generates the Prufer sequence that is delta beyond the current one.


\verb!nodes!\\
Returns the number of nodes in the tree.


\verb!prev(delta=1)!\\
Generates the Prufer sequence that is -delta before the current one.



\verb!prufer_rank()!\\
Computes the rank of a Prufer sequence.


\verb!prufer_repr!\\
Returns Prufer sequence for the Prufer object.


\verb!rank!\\
Returns the rank of the Prufer sequence.


\verb!size!\\
Return the number of possible trees of this Prufer object.



\verb!static to_prufer(tree, n)!\\
Return the Prufer sequence for a tree given as a list of edges where n is the number of nodes in the tree.



\verb!static to_tree(prufer)!\\
Return the tree (as a list of edges) of the given Prufer sequence.



\verb!tree_repr!\\
Returns the tree representation of the Prufer object.


\verb!unrank(rank, n)!\\
Finds the unranked Prufer sequence.


\section{Subset}
Path: \verb!sympy.combinatorics.subsets.Subset!\\
Represents a basic subset object.

\subsection{Methods}

\verb!bitlist_from_subset(subset, superset)!\\
Gets the bitlist corresponding to a subset.


\verb!cardinality!\\
Returns the number of all possible subsets.



\verb!iterate_binary(k)!\\
This is a helper function. It iterates over the binary subsets by k steps. This variable can be both positive or negative.



\verb!iterate_graycode(k)!\\
It performs k step overs to get the respective Gray codes.


\verb!next_binary()!\\
Generates the next binary ordered subset.


\verb!next_gray()!\\
Generates the next Gray code ordered subset.


\verb!next_lexicographic()!\\
Generates the next lexicographically ordered subset. NOT IMPLEMENTED

\verb!prev_binary()!\\
Generates the previous binary ordered subset.



\verb!prev_gray()!\\
Generates the previous Gray code ordered subset.



\verb!prev_lexicographic()!\\
Generates the previous lexicographically ordered subset. NOT IMPLEMENTED

\verb!rank_binary!\\
Computes the binary ordered rank.



\verb!rank_gray!\\
Computes the Gray code ranking of the subset.


\verb!rank_lexicographic!\\
Computes the lexicographic ranking of the subset.



\verb!size!\\
Gets the size of the subset.



\verb!subset!\\
Gets the subset represented by the current instance.


\verb!subset_from_bitlist(super_set, bitlist)!\\
Gets the subset defined by the \verb!bitlist!.



\verb!subset_indices(subset, superset)!\\
Return indices of subset in superset in a list;
the list is empty if all elements of \verb!subset! are not in \verb!superset!.


\verb!superset!\\
Gets the superset of the subset.


\verb!superset_size!\\
Returns the size of the superset.


\verb!unrank_binary(rank, superset)!\\
Gets the binary ordered subset of the specified rank.



\verb!unrank_gray(rank, superset)!\\
Gets the Gray code ordered subset of the specified rank.



\verb!subsets.ksubsets(superset, k)!\\
Finds the subsets of size \verb!k! in lexicographic order.


\section{Gray Code}
Path: \verb!sympy.combinatorics.graycode.GrayCode!
A Gray code is essentially a Hamiltonian walk on an n-dimensional
 cube with edge length of one. The vertices of the cube are
 represented by vectors whose values are binary.
 The Hamilton walk visits each vertex exactly once.

\subsection{Methods}

\verb!current!\\
Returns the currently referenced Gray code as a bit string.


\verb!generate_gray(**hints)!\\
Generates the sequence of bit vectors of a Gray Code.



\verb!n!\\
Returns the dimension of the Gray code.


\verb!next(delta=1)!\\
Returns the Gray code a distance delta (default = 1) from the current value in canonical order.



\verb!rank!\\
Ranks the Gray code.



\verb!selections!\\
Returns the number of bit vectors in the Gray code.



\verb!skip()!\\
Skips the bit generation.


\verb!unrank(n, rank)!\\
Unranks an n-bit sized Gray code of rank k.
This method exists so that a derivative GrayCode class can define its own code of a given rank.


\verb!graycode.random_bitstring(n)!\\
Generates a random bitlist of length n.


\verb!graycode.gray_to_bin(bin_list)!\\
Convert from Gray coding to binary coding.


\verb!graycode.bin_to_gray(bin_list)!\\
Convert from binary coding to gray coding.



\verb!graycode.get_subset_from_bitstring(super_set, bitstring)!\\
Gets the subset defined by the bitstring.



\verb!graycode.graycode_subsets(gray_code_set)!\\
Generates the subsets as enumerated by a Gray code.



\section{Named Groups}
Path: \verb!sympy.combinatorics.named_groups!\\

\subsection{Methods}

\verb!SymmetricGroup(n)!\\
Generates the symmetric group on \verb!n! elements as a permutation group.



\verb!CyclicGroup(n)!\\
Generates the cyclic group of order n as a permutation group.


\verb!DihedralGroup(n)!\\
Generates the dihedral group Dn as a permutation group.



\verb!AlternatingGroup(n)!\\
Generates the alternating group on \verb!n! elements as a permutation group.



\verb!AbelianGroup(*cyclic_orders)!\\
Returns the direct product of cyclic groups with the given orders.



\section{Utilities}
Path: \verb!sympy.combinatorics.util!

\subsection{Methods}
\verb!_base_ordering(base, degree)!\\
Order $\{0,1,\ldots,n\}$ so that base points come first and in order


\verb!_check_cycles_alt_sym(perm)!\\
Checks for cycles of prime length $p$ with $n/2 < p < n-2$.



\verb!_distribute_gens_by_base(base, gens)!\\
Distribute the group elements \verb!gens! by membership in basic stabilizers.


\verb!_handle_precomputed_bsgs(base, strong_gens,!\\
\verb!transversals=None, basic_orbits=None,!\\
\verb!strong_gens_distr=None)!\\
Calculate BSGS-related structures from those present.


\verb!_orbits_transversals_from_bsgs(base, strong_gens_distr,!\\
\verb!transversals_only=False)!\\
Compute basic orbits and transversals from a base and strong generating set.



\verb!_remove_gens(base, strong_gens,!
\verb!basic_orbits=None, strong_gens_distr=None)!\\
Remove redundant generators from a strong generating set.


\verb!_strip(g, base, orbits, transversals)!\\
Attempt to decompose a permutation using a (possibly partial) BSGS structure.


\verb!_strong_gens_from_distr(strong_gens_distr)!\\
Retrieve strong generating set from generators of basic stabilizers.



\section{Group Constructors}
Path: \verb!sympy.combinatorics.group_constructs!

\subsection{Method}
\verb!DirectProduct(*groups)!\\
Returns the direct product of several groups as a permutation group.




\section{Test Utilities}
Path: \verb!sympy.combinatorics.testutil!

\subsection{Methods}
\verb!_cmp_perm_lists(first, second)!\\
Compare two lists of permutations as sets.

\verb!_naive_list_centralizer(self, other)!\\


\verb!_verify_bsgs(group, base, gens)!\\
Verify the correctness of a base and strong generating set.

\verb!_verify_centralizer(group, arg, centr=None)!\\
Verify the centralizer of a group/set/element inside another group.

\verb!_verify_normal_closure(group, arg, closure=None)!\\



\rule{0.3\linewidth}{0.25pt}
\scriptsize\\
http://www.sympy.org/cheatsheets


\end{multicols}
\end{document}
